\documentclass[12pt]{article}

\oddsidemargin  -0.5 cm
\evensidemargin 0.0 cm
\textwidth      6.5in
\headheight     0.0in
\topmargin      -1 cm
\textheight=9.8in

\title{Alignment of the CMS Muon System \\ with the HIP Algorithm}
\author{Jim Pivarski}

\begin{document}
\maketitle

\section{Introduction to the HIP algorithm and the geometry of the muon system}

{\it (I copied the following from
https://twiki.cern.ch/twiki/bin/view/CMS/SWGuideMuonAlignAlgos.  The
CMS Note introduction won't be quite as short and technical--- it will
either be longer or contain less detail.)}

The HIP algorithm (Hits and Impact Points) is an iterative algorithm
for aligning detectors. It decouples the correlation between
track-fitting and detector alignment by alternating between the two,
hopefully converging to a solution that optimizes both. The alignment
step of the HIP algorithm is very simple: chambers are moved in such a
way as to make the weighted mean of their residuals distribution zero
before refitting. (A residual is the difference between the local
position of a hit and the track's extrapolation to the detector
surface, called the impact point.) In the one-dimensional case, an x
alignment correction is equal to the negative of the weighted mean of
the x residual distribution. In general, x and y residuals (if the
latter are available) are transformed to the 6-dimensional space of
parameters x, y, z, phix, phiy, and phiz (assuming all 6 parameters
are selected for alignment), through a 6x2 Jacobian matrix. The
weighted mean generalizes into a matrix equation involving a matrix
inversion that is at most 6x6, for every detector. Most of the
computational effort is in re-fitting tracks and producing updated
residuals distributions between each iteration. (See CMS NOTE
2006/018.)

The muon system forms a barrel and an endcap around the CMS detector
composed of 790 large (meter-scale) independently-alignable
sensors. Disks and wheels are the largest independently-alignable
structures (tens of meters), each of which holds dozens of
chambers. Alignment of the 6-12 layers and superlayers inside the
chambers must also be considered, though probably only once, with high
statistics. More detail about these issues and progress in solving
them will be available at SomePageThatDoesntExist?; the following is
an introduction to a software system that can compute these
alignments, given a dataset in CMSSW.

\section{Nominal procedure}

default, standard of comparison

10 pb$^{-1}$ of high $p_T$ tracks (from $Z$'s and $W$'s)

number of hits cuts on tracks, pull cut on hits

Initial misalignment: the Muon10InvPbScenario with residual layer
misalignments (to be determined from layer alignment studies)

chamber-by-chamber (CSC and DT) x, phiy, phiz float?  can we relax
this, letting y and phix float, too?  Not for DT station 4, of course...

globalMuon APE scheme: starts large, descends rapidly, converges
in a few iterations

standAloneMuon APE scheme: less steep, but relevant

\section{Accuracy and Precision}

output alignment parameter uncertainties (are they broken?  why
are they $\sim$4 times too big?): do they scale with residual
misalignment?

if not, is there any other way to identify chambers with poor
convergence?  do they have fewer hits?

\section{Performance issues}

How does this scale with number of tracks (convert to pb$^{-1}$ of $W$,
$Z$, and if possible, a generic muon sample with a cut)

How much computer time does this take?  (per iteration, and how
many iterations are necessary?)

\section{Special alignments}

\subsection{Quick disk alignment}

how much data do we need to do this if the
chambers and layers are also misaligned?  (With ideal chambers,
it's very quick! {\tt :)}

MC and MTCC (compare with known x = 7 mm, y = 1 mm in MTCC phase II)

\subsection{CSC Layer alignments}

what is the nominal procedure, what degrees
of freedom can we align, and how much data do we need to do it?

MC and MTCC

this will probably include Karoly's work: his plots with our results overlaid

\section{Beam halo alignment of CSC layers}

will we have this before data-taking?

New information from Karoly: no problem for inner ring, outer ring may
have 0.8 million muons, though heavily emphasizing the inner radius part

\section{Systematics studies}

\subsection{Dependence on tracker alignment}

How does the globalMuon strategy depend on tracker alignment?
(Initial studies suggest that the dependence is {\it very} weak.  Is
there anything wrong with my studies?)

New information: it looks like my procedure is working.  Dependence on
tracker alignment is probably {\it very weak!}  I need to pull this
together into a coherent story, with plots.

\subsection{Dependence on momentum}

How does resolution depend on the momentum of the input tracks?
Use $J/\psi$s to do an alignment if you have enough; otherwise, just
compare width of the residuals distributions.

\subsection{Dependence on fitting parameters}

What happens if we use different sets of fitting parameters?  I
don't see much difference yet, but I haven't tried dropping "y"
degrees of freedom.

\subsection{Correlation between alignment and calibration}

How well does alignment fare if we have a miscalibrated detector?

\subsection{``Dependence on tracking algorithm''}

\subsubsection{Uncertainty in material distribution}
Change the distribution of material in track-fitting but not
in SimHit generation to simulate incorrect material
description

\subsubsection{Uncertainty in magnetic field}

Same for magnetic field modeling

\subsubsection{Dependence on track charge}

How does alignment differ between positive and negative
tracks?  Can we cancel the effect by requiring equal
populations?

\subsubsection{Depdence on the actual algorithm used in tracking}

If I'm very adventurous, I may try plugging in different
tracking algorithms.  I don't know how easy that is.

\subsection{Background studies}

Align with realistic backgrounds from CSA07

What are the optimized track quality cuts?

\section{Conclusion}

It works very, very, very well.

\end{document}
